\documentclass{beamer}

\mode<presentation>
{
  \usetheme{Warsaw}
  % or ...

% to make stuff transparent instead of invisible
%  \setbeamercovered{transparent}
}


\usepackage[english]{babel}
\usepackage[latin1]{inputenc}
\usepackage{times}
\usepackage[T1]{fontenc}
% Or whatever. Note that the encoding and the font should match. If T1
% does not look nice, try deleting the line with the fontenc.


\title[SVM for Network AD]{Support Vector Machines for\\ Network Anomaly Detection}
\author{G.~Vormayr}
\institute{University of Technology Vienna}
\date[CNSE 2014]{Communication Networks SE, 2014}
\subject{Support Vector Machines for Network Anomaly Detection}


\pgfdeclareimage[height=0.5cm]{university-logo}{pics/TU_Logo_SW}
\logo{\pgfuseimage{university-logo}}



\AtBeginSubsection[]
{
  \begin{frame}<beamer>{Outline}
    \tableofcontents[currentsection,currentsubsection]
  \end{frame}
}

\begin{document}

\begin{frame}
  \titlepage
\end{frame}

\begin{frame}{Outline}
  \tableofcontents
\end{frame}


% Structuring a talk is a difficult task and the following structure
% may not be suitable. Here are some rules that apply for this
% solution: 

% - Talk about 30s to 2min per frame. So there should be between about
%   15 and 30 frames, all told.

% - A conference audience is likely to know very little of what you
%   are going to talk about. So *simplify*!
% - In a 20min talk, getting the main ideas across is hard
%   enough. Leave out details, even if it means being less precise than
%   you think necessary.
% - If you omit details that are vital to the proof/implementation,
%   just say so once. Everybody will be happy with that.

\section{Support Vector Machines}

\begin{frame}{Motivation}
    % Drawing of Machine
\end{frame}

\subsection{Inner Workings}

\begin{frame}{Features}{Describe the Objects}
\end{frame}

\begin{frame}{Feature Vectors}
\end{frame}

\begin{frame}{Linear Separation}
\end{frame}

\begin{frame}{Error Margin}
\end{frame}

\begin{frame}{Kernels}{Nonlinear Separation}
\end{frame}

\begin{frame}{One Class}
\end{frame}

\subsection{Step by Step Guide}

\begin{frame}{Feature Selection}
\end{frame}

\begin{frame}{Normalisation}
\end{frame}

\begin{frame}{Kernel Selection}
\end{frame}

\begin{frame}{Training \& Validation}{Find the Best Parameters}
\end{frame}

\begin{frame}{Cross-Validation}{The Better Training}
\end{frame}

\subsection*{Summary}
\begin{frame}{Summary}
\end{frame}

\section{Network Anomaly Detection}

\subsection{Challenges}

\begin{frame}{Training Data}
\end{frame}

\begin{frame}{Continuous Training}
\end{frame}

\begin{frame}{Error Rate}
\end{frame}


\subsection{Possibilities}

\begin{frame}{Detect Unknown Anomalies}
\end{frame}

\begin{frame}{Combine Different Sources}
\end{frame}

\begin{frame}{Performance}
\end{frame}


\section*{Summary}

\begin{frame}{Summary}

  % Keep the summary *very short*.
  \begin{itemize}
  \item
    The \alert{first main message} of your talk in one or two lines.
  \item
    The \alert{second main message} of your talk in one or two lines.
  \item
    Perhaps a \alert{third message}, but not more than that.
  \end{itemize}
  
  % The following outlook is optional.
  \vskip0pt plus.5fill
  \begin{itemize}
  \item
    Outlook
    \begin{itemize}
    \item
      Something you haven't solved.
    \item
      Something else you haven't solved.
    \end{itemize}
  \end{itemize}
\end{frame}



% All of the following is optional and typically not needed. 
\appendix
\section<presentation>*{\appendixname}
\subsection<presentation>*{For Further Reading}

\begin{frame}[allowframebreaks]
  \frametitle<presentation>{For Further Reading}
    
  \begin{thebibliography}{10}
    
  \beamertemplatebookbibitems
  % Start with overview books.

  \bibitem{Author1990}
    A.~Author.
    \newblock {\em Handbook of Everything}.
    \newblock Some Press, 1990.
 
    
  \beamertemplatearticlebibitems
  % Followed by interesting articles. Keep the list short. 

  \bibitem{Someone2000}
    S.~Someone.
    \newblock On this and that.
    \newblock {\em Journal of This and That}, 2(1):50--100,
    2000.
  \end{thebibliography}
\end{frame}

\end{document}


